\nonstopmode{}
\documentclass[a4paper]{book}
\usepackage[times,inconsolata,hyper]{Rd}
\usepackage{makeidx}
\makeatletter\@ifl@t@r\fmtversion{2018/04/01}{}{\usepackage[utf8]{inputenc}}\makeatother
% \usepackage{graphicx} % @USE GRAPHICX@
\makeindex{}
\begin{document}
\chapter*{}
\begin{center}
{\textbf{\huge Package `rosettaR'}}
\par\bigskip{\large \today}
\end{center}
\ifthenelse{\boolean{Rd@use@hyper}}{\hypersetup{pdftitle = {rosettaR: Turn Plain Language Statements Into RDF, CSV, or Anything}}}{}
\ifthenelse{\boolean{Rd@use@hyper}}{\hypersetup{pdfauthor = {Tim Alamenciak; Tarek Al Mustafa}}}{}
\begin{description}
\raggedright{}
\item[Type]\AsIs{Package}
\item[Title]\AsIs{Turn Plain Language Statements Into RDF, CSV, or Anything}
\item[Version]\AsIs{0.1.0}
\item[Description]\AsIs{Core functions to work with Rosetta Statements, which use
templates to convert plain language strings into RDF, CSV or other
representations. This package is designed to make the creation of knowledge
graphs easier.}
\item[License]\AsIs{MIT + file LICENSE}
\item[Encoding]\AsIs{UTF-8}
\item[LazyData]\AsIs{true}
\item[Roxygen]\AsIs{list(markdown = TRUE)}
\item[RoxygenNote]\AsIs{7.3.3}
\item[Imports]\AsIs{dplyr, jinjar, shiny, utils, reticulate, jsonlite, rlang,
digest}
\item[Depends]\AsIs{R (>= 3.5)}
\item[Suggests]\AsIs{knitr, rmarkdown, testthat (>= 3.0.0), shinytest2}
\item[VignetteBuilder]\AsIs{knitr}
\item[URL]\AsIs{}\url{https://timalamenciak.github.io/rosettaR/}\AsIs{}
\item[BugReports]\AsIs{}\url{https://github.com/timalamenciak/rosettaR/issues}\AsIs{}
\item[Config/testthat/edition]\AsIs{3}
\item[NeedsCompilation]\AsIs{no}
\item[Author]\AsIs{Tim Alamenciak [aut, cre],
Tarek Al Mustafa [aut]}
\item[Maintainer]\AsIs{Tim Alamenciak }\email{tim.alamenciak@gmail.com}\AsIs{}
\end{description}
\Rdcontents{Contents}
\HeaderA{add\_template}{Add a template string to the library}{add.Rul.template}
%
\begin{Description}
Add a template string to the library
\end{Description}
%
\begin{Usage}
\begin{verbatim}
add_template(library, template)
\end{verbatim}
\end{Usage}
%
\begin{Arguments}
\begin{ldescription}
\item[\code{library}] Input must be a library dataframe created by initLibrary()

\item[\code{template}] This should be one string template.
\end{ldescription}
\end{Arguments}
%
\begin{Value}
A new library data frame containing the template that was added.
\end{Value}
%
\begin{Examples}
\begin{ExampleCode}
templates <- init_library()
templates <- add_template(templates, apple_template)
\end{ExampleCode}
\end{Examples}
\HeaderA{apple\_csv}{Apple Output Template Example This is an example of an output template that reformats the statement as a CSV.}{apple.Rul.csv}
\keyword{datasets}{apple\_csv}
%
\begin{Description}
Apple Output Template Example
This is an example of an output template that reformats the
statement as a CSV.
\end{Description}
%
\begin{Usage}
\begin{verbatim}
apple_csv
\end{verbatim}
\end{Usage}
%
\begin{Format}
%
\begin{SubSection}{\code{apple\_csv}}

A string that contains Jinja syntax placeholders that match
the \code{apple\_template}.
\end{SubSection}

\end{Format}
%
\begin{References}
\url{https://arxiv.org/pdf/2407.20007}
\end{References}
\HeaderA{apple\_statement}{Apple Rosetta Statement Example This is an example of a statement that describes the weight of Apple X.}{apple.Rul.statement}
\keyword{datasets}{apple\_statement}
%
\begin{Description}
Apple Rosetta Statement Example
This is an example of a statement that describes the weight of Apple X.
\end{Description}
%
\begin{Usage}
\begin{verbatim}
apple_statement
\end{verbatim}
\end{Usage}
%
\begin{Format}
%
\begin{SubSection}{\code{apple\_statement}}

A string that specifies an object, a quality, a measurement and a unit.
\end{SubSection}

\end{Format}
%
\begin{References}
\url{https://arxiv.org/pdf/2407.20007}
\end{References}
\HeaderA{apple\_template}{Apple Rosetta Statement Example This is an example of a template to describe the weight of a particular apple.}{apple.Rul.template}
\keyword{datasets}{apple\_template}
%
\begin{Description}
Apple Rosetta Statement Example
This is an example of a template to describe the
weight of a particular apple.
\end{Description}
%
\begin{Usage}
\begin{verbatim}
apple_template
\end{verbatim}
\end{Usage}
%
\begin{Format}
%
\begin{SubSection}{\code{apple\_template}}

A string with slots for an object, a quality, a measurement and a unit.
\end{SubSection}

\end{Format}
%
\begin{References}
\url{https://arxiv.org/pdf/2407.20007}
\end{References}
\HeaderA{df\_to\_statements}{Change a data frame into a series of statements based on templates.}{df.Rul.to.Rul.statements}
%
\begin{Description}
Change a data frame into a series of statements based on templates.
\end{Description}
%
\begin{Usage}
\begin{verbatim}
df_to_statements(df, templates)
\end{verbatim}
\end{Usage}
%
\begin{Arguments}
\begin{ldescription}
\item[\code{df}] A data frame with headers that match the slots in a template. The
dataframe must have one column called "TemplateID" that contains the ID
of the template used to interpret the row.

\item[\code{templates}] A Rosetta Statement template library created by the
initLibrary function.
\end{ldescription}
\end{Arguments}
%
\begin{Value}
A data frame statement library containing the statements and
associated templates.
\end{Value}
%
\begin{Examples}
\begin{ExampleCode}
templates <- init_library()
templates <- add_template(templates, apple_template)
df <- data.frame(object = "Apple X",
                              quality = "weight", value="100.2",
                             unit="grams", TemplateID = "1")
statements <- df_to_statements(df,templates)
\end{ExampleCode}
\end{Examples}
\HeaderA{init\_library}{Load a set of Rosetta templates into a dataframe.}{init.Rul.library}
%
\begin{Description}
Load a set of Rosetta templates into a dataframe.
\end{Description}
%
\begin{Usage}
\begin{verbatim}
init_library(file = NA)
\end{verbatim}
\end{Usage}
%
\begin{Arguments}
\begin{ldescription}
\item[\code{file}] A CSV with the headers 'TemplateID' and 'templateText'.
Headers are case-sensitive and required. Other functions
like \code{df\_to\_statements()} and \code{add\_template()} will expect a data frame in the
format created by this function.
\end{ldescription}
\end{Arguments}
%
\begin{Value}
A dataframe containing either the loaded templates or the proper
headers for use with other functions.
\end{Value}
%
\begin{Examples}
\begin{ExampleCode}
#Initialize an empty library
templates <- init_library()

#Load a CSV library
apple_templates <- init_library(system.file("extdata", "apple_templates.csv",
 package="rosettaR"))
\end{ExampleCode}
\end{Examples}
\HeaderA{rosetta\_format}{Parse a Rosetta Statement}{rosetta.Rul.format}
%
\begin{Description}
Converts a plain language statement into a structured dataframe using
a template.
Supports variables \{\{ var \}\} and optional blocks [ ... ].
\end{Description}
%
\begin{Usage}
\begin{verbatim}
rosetta_format(s, in_template, out_template = "df")
\end{verbatim}
\end{Usage}
%
\begin{Arguments}
\begin{ldescription}
\item[\code{s}] The input statement string.

\item[\code{in\_template}] The Rosetta template string.

\item[\code{out\_template}] The output format ('df' for dataframe,
'rdf' for generic Turtle, or a Jinja string).
\end{ldescription}
\end{Arguments}
%
\begin{Value}
Either a data frame or the output template with values filled in from
statement and input template.
\end{Value}
%
\begin{Examples}
\begin{ExampleCode}
# example code
rosetta_format("Apple X weighs 235 grams", "{{ fruit }} weighs {{ value }} {{ unit}}")

\end{ExampleCode}
\end{Examples}
\HeaderA{rosetta\_match}{Match Multiple Statements Against a Template Library}{rosetta.Rul.match}
%
\begin{Description}
Iterates through a vector of statements and attempts to match them against
a library of templates. Returns a long-format dataframe.
\end{Description}
%
\begin{Usage}
\begin{verbatim}
rosetta_match(statements, templates)
\end{verbatim}
\end{Usage}
%
\begin{Arguments}
\begin{ldescription}
\item[\code{statements}] A character vector of plain language statements, OR a
dataframe containing a column of statements.

\item[\code{templates}] A dataframe created by \code{init\_library()}, containing
'TemplateID' and 'templateText' columns.
\end{ldescription}
\end{Arguments}
%
\begin{Value}
A data.frame in long format with columns: \code{statement\_id},
\code{statement\_text}, \code{template\_id}, variable, value.
\end{Value}
%
\begin{Examples}
\begin{ExampleCode}
## Not run: 
   # Vector Input
   statements <- c("Kitchener is in Canada", "Paris is in France")

   # Dataframe Input (e.g. from read.csv)
   # df <- read.csv("my_statements.csv")

   templates <- init_library(system.file("extdata/apple_templates.csv",
    package="rosettaR"))
   results <- rosetta_match(statements, templates)

## End(Not run)
\end{ExampleCode}
\end{Examples}
\HeaderA{rosetta\_triplify}{Batch Convert Statements to RSO RDF}{rosetta.Rul.triplify}
%
\begin{Description}
Converts a list of statements into a single, cohesive Turtle (TTL) string
using the Rosetta Statement Ontology (RSO).
Handles prefix management automatically.
\end{Description}
%
\begin{Usage}
\begin{verbatim}
rosetta_triplify(statements, templates)
\end{verbatim}
\end{Usage}
%
\begin{Arguments}
\begin{ldescription}
\item[\code{statements}] A character vector of plain language statements.

\item[\code{templates}] A dataframe created by \code{init\_library()}.
\end{ldescription}
\end{Arguments}
%
\begin{Value}
A single character string containing the full RDF document.
\end{Value}
%
\begin{Examples}
\begin{ExampleCode}
## Not run: 
  # (Use the same setup as above)
  ttl <- rosetta_triplify(statements, templates)
  cat(ttl)

## End(Not run)
\end{ExampleCode}
\end{Examples}
\HeaderA{rosetta\_validate}{Validate a data frame against a LinkML schema}{rosetta.Rul.validate}
%
\begin{Description}
Validate a data frame against a LinkML schema
\end{Description}
%
\begin{Usage}
\begin{verbatim}
rosetta_validate(data, schema, target_class = NULL)
\end{verbatim}
\end{Usage}
%
\begin{Arguments}
\begin{ldescription}
\item[\code{data}] A data frame (e.g. from `rosetta\_format) or list.

\item[\code{schema}] Path to the LinkML schema YAML file.

\item[\code{target\_class}] The class in the schema to validate against.
\end{ldescription}
\end{Arguments}
%
\begin{Value}
A list containing \code{ok} (boolean) and \code{issues} (list of errors).
\end{Value}
%
\begin{Examples}
\begin{ExampleCode}
## Not run: 
  # Create a YAML schema for this example.
  schema_yaml <- "id: https://example.org/apple-schema
  name: AppleSchema
   imports:
   - linkml:types
   default_range: string
 classes:
   AppleObservation:
    attributes:
      object:
        required: true
      value:
        range: float  # Value must be a number!
      unit:
        range: string"

# Save to a temporary file for this example
schema_file <- tempfile(fileext = ".yaml")
writeLines(schema_yaml, schema_file)# (Use the same setup as above)

good_data <- data.frame(object = "Apple A", value = 150.5, unit = "g")
res_good <- rosetta_validate(good_data, schema_file,
  target_class = "AppleObservation")
print(paste("Good Data is Valid:", res_good$ok))

## End(Not run)
\end{ExampleCode}
\end{Examples}
\HeaderA{run\_rosetta\_ui}{Launch the RosettaR UI}{run.Rul.rosetta.Rul.ui}
%
\begin{Description}
Opens a Shiny application to interactively test Rosetta templates
and validation.
\end{Description}
%
\begin{Usage}
\begin{verbatim}
run_rosetta_ui()
\end{verbatim}
\end{Usage}
%
\begin{Value}
A character string containing the processed text
(if \code{out\_template} is text),
or a dataframe (if \code{out\_template} is "df").
\end{Value}
%
\begin{Examples}
\begin{ExampleCode}
## Not run: 
run_rosetta_ui()

## End(Not run)

\end{ExampleCode}
\end{Examples}
\printindex{}
\end{document}
